% !TEX root = ../index.tex

While Artificial Intelligence (AI)'s capabilities are becoming more powerful than ever, 
they demand substantial computational resources, raising critical concerns about cost, 
scalability, and environmental impact, which is already becoming a significant issue, 
for example with the data centres and Graphics Processing Unit (GPU) clusters.
This brings challenges for small and medium-sized enterprises (SMEs), which often lack the infrastructure 
and capital of their larger counterparts, hindering their ability to adopt AI efficiently. 
This thesis aims to cover this gap by proposing a Proof-of-Concept (PoC) benchmarking framework 
designed to assess AI efficiency across four key dimensions: accuracy, runtime, cost, and energy consumption. 
The framework is developed through a comprehensive literature review of existing benchmarking tools and a critical 
evaluation of modern AI hardware architectures, including GPUs, specialised accelerators, and emerging data centre infrastructures.
As a practical implementation, this research introduces Project HANSAL (Hybrid AI for Next-gen: Sustainable, Affordable, 
and Lightweight). This concept is applied to the framework to provide SMEs with insights for their domain-specific 
business cases by measuring their AI systems that strategically balance high performance with resource sustainability.
Providing a structured framework and reference guide for resource optimisation in AI, this thesis aims to offer both 
academic and industrial values by supporting energy-aware, cost-efficient AI adoption. While it focuses on its 
applicability for SMEs, it touches on the possibility for larger corporations to adopt this concept and framework, 
based on their business needs and counterparts.

